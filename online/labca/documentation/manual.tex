\documentclass{article}
%\usepackage[T1]{fontenc}
\usepackage{textcomp}
\usepackage{bookman}
\usepackage{color}
\usepackage{graphics}
\usepackage{html, makeidx}

\newcommand{\sca}{\ita{labCA}}
\newcommand{\scilab}{\ita{scilab}}
\newcommand{\matlab}{\ita{matlab}}
\newcommand{\windoze}{\ita{win32}}
\newcommand{\linux}{\ita{linux}}
\newcommand{\solaris}{\ita{solaris}}
\newcommand{\ezca}{\ita{ezca}}
\newcommand{\com}[1]{{\tt #1}}
\newcommand{\NAN}{\mbox{NaN}}
\newcommand{\pbrk}{\pagebreak[3]}
\newcommand{\pbrkf}{\pagebreak}
\newcommand{\comref}[2]{\hyperref[ref]{\com{#1}}{\com{#1} (see }{)}{#2}}

\newcommand{\ita}[1]{\emph{#1}}

\newcommand{\m}{$m$}
\newcommand{\mhack}{$m$} %latex2html didnt properly renewcommand \m inside \PVITEM ???
\newcommand{\n}{$n$}
\newcommand{\mxn}{$m\times n$}
\newcommand{\mxl}{$m\times 1$}
\newcommand{\lxn}{$1\times n$}
\newcommand{\calimit}{$\approx 16$kB}
\newcommand{\sevrange}{$0..3$}


\begin{htmlonly}
\renewcommand{\m}{\ita{m}}
\newcommand{\mhack}{\ita{m}} %latex2html didnt properly renewcommand \m inside \PVITEM ???
\renewcommand{\n}{\ita{n}}
\renewcommand{\mxn}{\ita{m x n}}
\renewcommand{\mxl}{\ita{m x 1}}
\renewcommand{\lxn}{\ita{1 x n}}
\renewcommand{\calimit}{\~{} 16kB}
\renewcommand{\sevrange}{0..3}
\renewcommand{\pbrk}{}
\renewcommand{\pbrkf}{}
\end{htmlonly}


\newcommand{\PVITEM}{
\item[pvs] Column vector (in matlab: \mxl{} \ita{cell-} matrix)
of \mhack{} strings.
}


\title{\sca{} --- An EPICS Channel Access Interface for \scilab{} and \matlab{}}
\author{Till Straumann $<$straumanATslac.stanford.edu$>$, 2003}
\date{May 23, 2007}

\begin{document}
\maketitle
{\hspace*{\fill}{\small\verb$Id: manual.tex,v 1.38 2007/06/05 05:00:12 guest Exp $}\hspace*{\fill}}
\section{Introduction}
The \sca{} package provides an interface to the
EPICS channel access client library which can be
integrated with the \scilab{} or \matlab{} applications.
Both, \scilab{} and \matlab{} feature an API for interfacing
to user binaries written in a programming language such
as C or Fortran. \sca{} properly wraps the essential
channel access routines and makes them accessible
from the \scilab%
\footnote{throughout this text, references to \scilab{}
usually mean \scilab{} or \matlab.}
command line.

\sca{} actually uses an extra layer, the \ezca{} library
which transparently manages and caches channel access 
connections. A modified version of \ezca{} comes with
\sca, adding thread-safety and hence EPICS 3.14 fitness.

As of \sca{} version 3 further improvements to \ezca{}
have been made that exploit features of the multi-threaded
CA library (EPICS 3.14 only) in order to speed up response time.
Earlier versions always handed control to \ezca{} in multiples of the
\sca{} \ita{timeout}, i.e., even if data from a channel
were available quicker the library would wait until the
next timeout quantum expired. Since version 3 a \sca{}
call returns immediately after the underlying request
completes.

A very convenient feature of \sca{} is the ability
to execute \ezca{} calls on groups of PVs, simply by
passing the respective \sca{} routine a column vector
of PV names.

\sca{} has been tested with EPICS 3.13%
\footnote{Support for 3.13 has been dropped as of \sca{} version 3}%
,
3.14, \scilab-2.7 .. \scilab-4.1,
\matlab-6.5, \matlab-7 on \ita{linux}, \ita{solaris} and
\windoze. Note that while some of these combinations have been
tested and been known working in the past, only the latest
versions of the respective components have been tested
and verified to build and run successfully with the
current version of \sca{}. Modifications to the \com{Makefile}s
might be necessary to build older versions.

\vspace*{\fill}
\pbrk
\section{Supported EZCA Calls}
\sca{} implements an interface to almost all public
\ezca{} routines\footnote{%
the \matlab{} implementation may still lack some of the
more esoteric commands}. Note that the arguments and
return values do not exactly correspond to the respective
\ezca{} originals but have been adapted to make sense
in the  \scilab{}%
\footnote{throughout this text, references to \scilab{}
usually mean \scilab{} or \matlab.}
environment.

\subsection{Common Arguments and Return Values}
\subsubsection{PV Argument}
All \sca{} calls take a \ita{PV} argument identifying
the EPICS process variables the user wants to connect to.
PVs are plain ASCII strings. \sca{} is capable of handling
multiple PVs in a single call; they are simply passed as a
column vector:
\begin{verbatim}
pvs = [ 'PVa'; 'b'; 'anotherone' ]
\end{verbatim}
Because \matlab{} doesn't allow the rows of
a string vector to be of different size, the \matlab{}
wrapper expects a \ita{cell-} array of strings:
\begin{verbatim}
pvs = { 'PVa'; 'b'; 'anotherone' }
\end{verbatim}

All channel access activities for the PVs
passed to a single \sca{} call are batched together and
completion of the batch is awaited before returning from
the \sca{} call. Consider the following example:%
\footnote{
In \matlab{}, the square brackets (``[]'') must be replaced
by curly braces (``\{\}'').}
\begin{verbatim}
lcaPut( 'trigger', 1 ) \\
data=lcaGet( ['sensor1' ; 'sens2'] );
\end{verbatim}
\begin{itemize}
%
\item It is guaranteed that writing the ``trigger''
completes (on the CA server) prior to reading the
sensors.%%
\footnote{If the remote sensors have finite
processing time, the subsequent CA read may still
get old data --- depending on the device support
etc.; this is beyond the scope of channel access, however.}
%
\item Reading the two sensors is done in ``parallel'' ---
the exact order is unspecified. After the command sequence
(successfully) completes, all the data are valid.
\end{itemize}

\subsubsection{Timestamp Format}
\label{tsformat}
Channel access timestamps are ``POSIX struct timespec''
compliant, i.e. they provide the number of nanoseconds
expired since 00:00:00 UTC, January 1, 1970. \sca{} 
translates the timestamps into \ita{complex numbers}
with the seconds (\com{tv\_sec} member) and nanoseconds
(\com{tv\_nsec}) in the real and imaginary parts, respectively.
This makes it easy to extract the seconds while still maintaining
full accuracy.

\subsection{Error Handling}
\label{errorhandling}
All errors%
\footnote{As of version 3; earlier versions didn't consistently
``throw'' all errors so that they could be caught by the
\com{try-catch-end} mechanism but would merely print messages
when encountering some minor errors. Also, versions earlier than 3
would not report error IDs/messages to \com{lasterror} nor
implement the \comref{lcaLastError}{lcalasterror} routine.}
encountered during execution of \sca{} calls result in the
call being aborted, a message being printed to the console
and an error status being recorded which
can be retrieved using \scilab{}'s \com{lasterror} command.
The recommended method for handling errors is \scilab{}'s
\mbox{\com{try -- catch -- end}} construct:
\begin{verbatim}
	try
		val = lcaGet(pvvector)
	catch
		err   = lasterror()
		// additional information is provided
		// by the 'lcaLastError' routine
		stats = lcaLastError()
		// handle error here
	end
\end{verbatim}
Many \sca{} calls can handle multiple PVs at once
and underlying CA operations may fail for a subset of PVs only.
However, \com{lasterror} only supports reporting a single error
status. Therefore the \comref{lcaLastError}{lcalasterror} routine
was implemented: if a \sca{} call fails for a subset of PVs then
a subsequent call to \com{lcaLastError} returns a column vector
of status values for the individual PVs.
The error codes are shown in
\hyperref{Table~}{Table~}{}{tbl:errors}%
\begin{htmlonly}
\ref{tbl:errors}%
\end{htmlonly}
.

\begin{table}[h]
\begin{tabular}{r|l|p{5.5cm}}
	\#                   &      Matlab Error ID         & Comment \\
	\hline
	0                    & \com{labca:unexpectedOK}     & No error \\
    1                    & \com{labca:invalidArg}       & Invalid argument \\
    2                    & \com{labca:noMemory}         & Not enough memory \\
    3                    & \com{labca:channelAccessFail}& Underlying CA operation failed \\
    4                    & \com{labca:udfCaReq}         & Item(s) requested undefined for its/their native data type\\
    5                    & \com{labca:notConnected}     & Channel currently disconnected\\
    6                    & \com{labca:timedOut}         & No response in time \\
    7                    & \com{labca:inGroup}          & Currently in a EZCA group\\
    8                    & \com{labca:notInGroup}       & Currently not in a EZCA group\\
    9                    & \com{labca:usrAbort}         & EZCA call aborted by user (Ctrl-C) \\
    20                   & \com{labca:noMonitor}        & No monitor for PV/type found\\
    21                   & \com{labca:noChannel}        & No channel for PV name found\\
\end{tabular}
\caption{~\sca{} error codes. Numerical codes (\scilab{} \com{lasterror} and
\com{lcaLastError()}) and corresponding \matlab{} error ``ID''s
(as returned by \matlab{} \com{lasterror}).}
\label{tbl:errors}
\end{table}


\vspace*{\fill}
\pbrk
\subsection{lcaGet}
\label{lcaget}
\subsubsection{Calling Syntax}
\begin{verbatim}
[value, timestamp] = lcaGet(pvs, nmax, type)
\end{verbatim}
\subsubsection{Description}
Read a number of \m{} PVs, which may be scalars or arrays of
different dimensions. The result is converted into a \mxn{}
matrix. The number of columns, \n, is automatically assigned
to fit the largest array among the \m{} PVs.  PVs with less than
\n{} elements have their excess elements in the result matrix
filled with \NAN.

If all PVs are of native type DBF\_STRING or DBF\_ENUM, the
values are returned as character strings; otherwise,
all values are converted into double precision numbers.
Explicit type conversion into strings can be enforced by submitting
the `type' argument described below.

\subsubsection{Parameters}
\begin{description}
\PVITEM
%
%
\item[nmax] \label{nmaxarg}
(\ita{optional argument}) Maximum number of elements
(per PV) to retrieve (i.e. limit the number of columns of \com{value} to
\ita{nmax}).
If set to 0 (default), all elements are fetched
and the number of columns, \n, in the result matrix is set to the
maximum number of elements among the PVs. The option is useful
to limit the transfer time of large waveforms (unfortunately,
CA does not return the valid elements (``NORD'') of
an array only --- it always ships \ita{all} elements).
%
%
\item[type] \label{typearg}
(\ita{optional argument}) A string specifying the
data type to be used for the channel access data transfer. Note
that unless the PVs are of native ``string'' type or conversion to
``string'' is enforced explicitely (type = \com{char}), \sca{}
always converts the data to ``double'' locally.

It can be desirable, however, to use a different data type for the
transfer because by default CA transfers are limited to \calimit.
Legal values for \com{type}{} are \com{byte}, \com{short}, \com{long},
\com{float}, \com{double}, \com{native} or \com{char} (strings).
There should rarely be a need for using anything other than \com{native},
the default, which directs CA to use the same type for transfer as the
data are stored on the server.

Occasionally, conversion to \com{char} can be useful: retrieve a number
of PVs as strings, i.e. let the CA server convert them to strings
(if the PVs are not native strings already) and transfer them.

If multiple PVs are requested, either none or all must be of native
DBF\_STRING or DBF\_ENUM type unless explicit conversion to \com{char}
is enforced by specifying this argument.

Note that while \com{native}{} might result in different types
being used for different PVs, it is currently not possible to
explicitely request different types for individual PVs (i.e.
\com{type}{} can't be a vector).
%
%
\item[value] The \mxn{} result matrix. \n{} is automatically
assigned to accomodate the PV with the most elements. If the \com{nmax}{}
argument is given and is nonzero but less than the automatically
determined \n, then \n{} is clipped to \com{nmax}.
Excess elements of PVs with less than \n{} elements are filled with
\NAN{} values.

The result is either a `double' or a (\matlab: \ita{cell-}) `string'
matrix (if all PVs are of native string type or explicit conversion
was requested by setting the `type' argument to `\com{char}').

\sca{} checks the channel access severity of the retrieved PVs and
fills the rows corresponding to \ita{INVALID} PVs with \NAN%
\footnote{
Actually, all fields of an EPICS database record share a common
severity, (which itself is a field/PV --- the .SEVR field). However,
the \ita{INVALID} status actually only applies to the .VAL field
of a record --- other fields (e.g. .EGU) may still hold meaningful data.
Consequently, \ita{INVALID} PVs are returned as \NAN{} only if
they refer to a record's .VAL field.
}. In addition, warning
messages are printed to the console if a PV's alarm status exceeds
a \hyperref[ref]{configurable threshold}{configurable threshold (see }{ )}{swarnlevel}.
The refusal to read PVs with \ita{INVALID} severity can be tuned using
the \com{lcaSetSeverityWarnLevel} call as well.
%
%
\item[timestamp] (\ita{optional result}) A \mxl{}
column vector of \ita{complex} numbers holding the CA timestamps of
the requested PVs. The timestamps count the number of seconds (real part)
and fractional nanoseconds (imaginary part) elapsed since
00:00:00 UTC, Jan. 1, 1970.
\end{description}
\subsubsection{Examples}
\begin{verbatim}
// read a PV
    lcaGet( 'thepv' )
// read multiple PVs along with their EPICS timestamps
    [ vals, tstamps] = lcaGet( [ 'aPV' ; 'anotherPV' ] )
// read an 'ENUM/STRING'
    lcaGet( 'thepv.SCAN' )
// read an 'ENUM/STRING' as a number (server converts)
    lcaGet( 'thepv.SCAN', 0, 'float' )
// enforce reading all PVs as strings (server converts)
// NOTE: necessary if native num/nonnum types are mixed
    lcaGet( [ 'apv.SCAN'; 'numericalPV' ] , 0, 'char' )
// limit reading a waveform to its NORD elements
    nord = lcaGet( 'waveform.NORD' )
	if nord > 0 then
      lcaGet( 'waveform', nord )
	end
\end{verbatim}

\pbrk
\subsection{lcaPut}
\label{lcaput}
\subsubsection{Calling Syntax}
\begin{verbatim}
lcaPut(pvs, value, type)
\end{verbatim}
\subsubsection{Description}
Write to a number of PVs which may be scalars or arrays of different
dimensions. It is possible to write the same value to a collection of PVs.
\subsubsection{Parameters}
\begin{description}
\PVITEM
%
%
\item[value] \mxn{} matrix or \lxn{} row vector of values
to be written to the PVs. If there is only a single row in \com{value}{}
it is written to all \m{} PVs. \com{value} may be a matrix of ``double''
precision numbers or  a (\matlab: \ita{cell-}) matrix of strings (in
which case the values are transferred as strings and converted by the
CA server to the native type --- this is particularly useful for
DBF\_ENUM / ``menu'' type PVs).

It is possible to write less than \n{} elements --- \sca{} scans all rows
for \NAN{} values and only transfers up to the last non-\NAN{} element in each
row.
%
%
\item[type] (\ita{optional argument}) A string specifying the
data type to be used for the channel access data transfer. Note
that \sca{} always converts numerical data from ``double'' locally.

It can be desirable, to use a different data type for the
transfer because by default CA transfers are limited to \calimit.
Legal values for \com{type}{} are \com{byte}, \com{short}, \com{long},
\com{float}, \com{double}, \com{char} or \com{native}. There should rarely
be a need for using anything other than \com{native}, the default,
which directs CA to use the same type for transfer as the data are
stored on the server. If \com{value} is a string matrix, \com{type}
is automatically set to \com{char}.

Note that while \com{native}{} might result in different types
being used for different PVs, it is currently not possible to
explicitely request different types for individual PVs (i.e. 
\com{type}{} cannot be a vector).
%
\end{description}

\subsubsection{Examples}
\begin{verbatim}
// write a PV
    lcaPut( 'thepv', 1.234 )
// write as a string (server converts)
    lcaPut( 'thepv', '1.234' )
// write/transfer as a short integer (server converts)
    lcaPut( 'thepv', 12, 'short' )
// write multiple PVs (use { } on matlab)
    lcaPut( [ 'pvA'; 'pvB' ], [ 'a'; 'b' ] );
// write array PV
    lcaPut( 'thepv' , [ 1, 2, 3, 4 ] )
// write same value to a group of PVs (string
// concatenation differs on matlab)
    lcaPut( [ 'pvA'; 'pvB' ] + '.SCAN', '1 second' )
// write array and scalar PV (using NaN as a delimiter)
    tab = [ 1, 2, 3, 4 ;   5, %nan, 0, 0 ]
	lcaPut( [ 'arrayPV'; 'scalarPV' ], tab )
\end{verbatim}

\vspace*{\fill}
\pbrkf
\subsection{lcaPutNoWait}
\label{lcaputnowait}
\subsubsection{Calling Syntax}
\begin{verbatim}
lcaPutNoWait(pvs, value, type)
\end{verbatim}
\subsubsection{Description}
\com{lcaPutNoWait} is a variant of \com{lcaPut} that does {\em not wait}
for the channel access put request to complete on the server prior to
returning control to the command line.
This call can be useful to set PVs that are known to take a long
or indefinite time to complete processing, e.g., arming a waveform record
which is triggered by a hardware event in the future or starting a stepper
motor.

\subsubsection{Parameters}
See \com{lcaPut}.

\vspace*{\fill}
\pbrk
\subsection{lcaGetNelem}
\subsubsection{Calling Syntax}
\begin{verbatim}
numberOfElements = lcaGetNelem(pvs)
\end{verbatim}
\subsubsection{Description}
Retrieve the element count of a number of PVs. Note that this is not
necessarily the number of \ita{valid} elements (e.g. the actual number
of values read from a device into a waveform) but the maximum number
of elements a PV can hold.
\subsubsection{Parameters}
\begin{description}
\PVITEM
%%
\item[numberOfElements] \mxl{} column vector of the PV's number
of elements (``array dimension'').
\end{description}

\pbrk
\subsection{lcaSetMonitor}
\label{lcasetmonitor}
\subsubsection{Calling Syntax}
\begin{verbatim}
lcaSetMonitor(pvs, nmax, type)
\end{verbatim}
\subsubsection{Description}
Set a ``monitor'' on a set of PVs. Monitored PVs are automatically retrieved
every time their value or status changes. 
Monitors are especially useful under EPICS-3.14 which supports
multiple threads. EPICS-3.14 transparently reads monitored PVs as needed.
Older, single threaded versions of EPICS require periodic calls to \sca{}
e.g., to \comref{lcaDelay}{lcadelay}, in order to allow \sca{} to handle
monitors.

Use the \comref{lcaNewMonitorValue}{lcanewmonitorvalue} call to check
monitor status (local flag) or \comref{lcaNewMonitorWait}{lcanewmonitorwait}
to wait for new data to become available (since last \com{lcaGet} or \com{lcaSetMonitor}).
If new data are available, they are retrieved using
the ordinary \comref{lcaGet}{lcaget} call.

Note the difference between polling and monitoring a PV in combination with
polling the local monitor status flag (\comref{lcaNewMonitorValue}{lcanewmonitorvalue}).
In the first case, remote data are fetched
on every polling cycle whereas in the second case, data are transferred only
when they change.  Also, in the monitored case, \com{lcaGet} reads from a local
buffer rather than from the network. It is most convenient however to wait
for monitored data to arrive using \comref{lcaNewMonitorWait}{lcanewmonitorwait}
rather than polling.

There is currently no possibility to selectively remove a monitor. Use
the \comref{lcaClear}{lcaclear} call to disconnect a channel and as a side-effect,
remove all monitors on that channel. Future access to a cleared channel
simply reestablishes a connection (but no monitors).
\subsubsection{Parameters}
\begin{description}
\PVITEM
\item[nmax]
(\ita{optional argument}) Maximum number of elements
(per PV) to monitor/retrieve. If set to 0 (default), all elements are fetched.
See \hyperref[ref]{here}{(}{)}{nmaxarg} for more information.

Note that a subsequent \comref{lcaGet}{lcaget} must specify a \com{nmax}
argument equal or less than the number given to \com{lcaSetMonitor} ---
otherwise the \com{lcaGet} operation results in fetching a new
set of data from the server because the \com{lcaGet} request cannot
be satisfied using the copy locally cached by the monitor-thread.
\item[type]
(\ita{optional argument}) A string specifying the
data type to be used for the channel access data transfer. The
native type is used by default.
See \hyperref[ref]{here}{(}{)}{typearg} for more information.

The type specified for the subsequent \com{lcaGet} for retrieving
the data should match the monitor's data type. Otherwise, \com{lcaGet}
will fetch a new copy from the server instead of using the data that
was already transferred as a result of the monitoring.
\end{description}
\subsubsection{Examples}
\begin{verbatim}
lcaSetMonitor('PV')
// monitor 'PV'. Reduce network traffic by just have the
// library retrieve the first 20 elements. Use DBR_SHORT
// for transfer.
lcaSetMonitor('PV', 20, 's')
\end{verbatim}

\pbrk
\subsection{lcaNewMonitorWait}
\label{lcanewmonitorwait}
\subsubsection{Calling Syntax}
\begin{verbatim}
lcaNewMonitorValue(pvs, type)
\end{verbatim}
\subsubsection{Description}
Similar to \comref{lcaNewMonitorValue}{lcanewmonitorvalue} but instead of
returning the status of monitored PVs this routine blocks until all PVs
have fresh data available (e.g., due to initial connection or changes in
value and/or severity status).
Reading the actual data must be done using \comref{lcaGet}{lcaget}.
\subsubsection{Parameters}
\begin{description}
\PVITEM
\item[type]
(\ita{optional argument}) A string specifying the
data type to be used for the channel access data transfer. The
native type is used by default.
See \hyperref[ref]{here}{(}{)}{typearg} for more information.

Note that monitors are specific to a particular data type and therefore
\com{lcaNewMonitorWait} will only report the status for a monitor that had been established
(by \comref{lcaSetMonitor}{lcasetmonitor}) with a matching type.
Using the ``native'' type, which is the default, for both calls
satisfies this condition.
\end{description}
\subsubsection{Examples}
\begin{verbatim}
try lcaNewMonitorWait(pvs)
	vals = lcaGet(pvs)
catch
	errs = lcaLastError()
	handleErrors(errs)
end
\end{verbatim}

\pbrk
\subsection{lcaNewMonitorValue}
\label{lcanewmonitorvalue}
\subsubsection{Calling Syntax}
\begin{verbatim}
[flags] = lcaNewMonitorValue(pvs, type)
\end{verbatim}
\subsubsection{Description}
Check if monitored PVs need to be read, i.e, if fresh data are available (e.g., due to
initial connection or changes in value and/or severity status). Reading the actual
data must be done using \comref{lcaGet}{lcaget}.
\subsubsection{Parameters}
\begin{description}
\PVITEM
\item[type]
(\ita{optional argument}) A string specifying the
data type to be used for the channel access data transfer. The
native type is used by default.
See \hyperref[ref]{here}{(}{)}{typearg} for more information.

Note that monitors are specific to a particular data type and therefore
\com{lcaNewMonitorValue} will only report the status for a monitor that had been established
(by \comref{lcaSetMonitor}{lcasetmonitor}) with a matching type.
Using the ``native'' type, which is the default, for both calls
satisfies this condition.
\item[flags]
Column vector of flag values. A value of zero indicates that no new data are available --
the monitored PV has not changed since it was last read (the data, that is, {\em not the flag}).
A value of one indicates that new data are available for reading (\com{lcaGet}).

{\em NOTE: As of \sca{} version 3 the \com{flags} no longer report error conditions.
Errors are now reported in the 
\hyperref{standard way}{standard way (see~}{)}{errorhandling},
i.e., by aborting the \sca{} call.
Errors can be caught by the standard \scilab{} \com{try}-\com{catch}-\com{end}
construct. The \comref{lcaLastError}{lcalasterror} routine can be used to obtain
status information for individual channels if \com{lcaNewMonitorValue} fails on
a vector of PVs.}

See also \comref{lcaNewMonitorWait}{lcanewmonitorwait}.
\end{description}
\subsubsection{Examples}
\begin{verbatim}
try and(lcaNewMonitorValue(pvvec))
	vals = lcaGet(pvvec)
catch
    errs = lcaLastError()
	handleErrs(errs)
end
\end{verbatim}

\pbrk
\subsection{lcaClear}
\label{lcaclear}
\subsubsection{Calling Syntax}
\begin{verbatim}
lcaClear(pvs)
\end{verbatim}
\subsubsection{Description}
Clear / release (disconnect) channels. This is particularly useful with
EPICS 3.14 to clean up invalid PVs (e.g., due to typos). Nonexisting PVs are
continuously searched for by a CA background task which may result
in cluttered IOC consoles and resource consumption.
All monitors on the target channel(s) are cancelled/released as a
consequence of this call.
\subsubsection{Parameters}
\begin{description}
\PVITEM
Alternatively, \com{lcaClear} may be called with {\em no} rhs argument
thus clearing {\em all} channels (and monitors).
\end{description}
\subsubsection{Examples}
\begin{verbatim}
\\ clear a number of channels
  lcaClear( ['aUseless_PV'; 'misTyppedPV' ] )
\\ purge all channels (dont use parenthesis in matlab)
  lcaClear()
\end{verbatim}

\vspace*{\fill}
\pbrkf
\subsection{lcaLastError}
\label{lcalasterror}
\subsubsection{Calling Syntax}
\begin{verbatim}
[err_status] = lcaLastError()
\end{verbatim}
\subsubsection{Description}
This routine is a simple extension to \scilab{}'s \com{lasterror}
which only allows a single error to be reported. If \sca{} encounters
an error of general nature then \com{lasterror} is sufficient
and \com{lcaLastError()} reports redundant/identical information.
However, if a \sca{} operation only fails on a subset of a vector
of PVs then \com{lcaLastError()} returns an error code for each individual
PV (as a \mxl{} vector) so that failing channels can be identified.

The error reported by \com{lasterror} corresponds to the first error
found in the \com{err\_status} vector.

Note that (matching \com{lasterror}'s semantics) the recorded
errors are {\it not cleared by a successful \sca{} operation}.
Hence, the status returned by \com{lcaLastError()} is only defined
after an error occurred and the routine is intended to be used
from the \com{catch} section of a \mbox{\com{try -- catch -- end}}
construct.

\subsubsection{Parameters}
\begin{description}
\item[err\_status] \mxl{} column vector of
\hyperref{status codes}{(see~}{)}{errorhandling}
for each PV of the last failing \sca{} call or a scalar.
Note that this routine can return a scalar even if
the last operation involved multiple PVs if the
error was of general nature (e.g., ``invalid argument'').
In this case the scalar is identical to the error reported
by \scilab{}'s \com{lasterror}.
\end{description}
\subsubsection{Examples}
\begin{verbatim}
try
  // lcaXXX command goes here
catch
  errors = lcaLastError()
  // errors holds status vector or single status code
  // depending on command, error cause and number of PVs.
end
\end{verbatim}

\vspace*{\fill}
\pbrkf
\subsection{lcaGetControlLimits}
\subsubsection{Calling Syntax}
\begin{verbatim}
[lowLimit, hiLimit] = lcaGetControlLimits(pvs)
\end{verbatim}
\subsubsection{Description}
Retrieve the control limits associated with a number of PVs.
\subsubsection{Parameters}
\begin{description}
\PVITEM
%%
\item[lowLimit] \mxl{} column vector of the PV's low control limit.
\item[hiLimit]  \mxl{} column vector of the PV's high control limit.
\end{description}


\vspace*{\fill}
\pbrk
\subsection{lcaGetGraphicLimits}
\subsubsection{Calling Syntax}
\begin{verbatim}
[lowLimit, hiLimit] = lcaGetGraphicLimits(pvs)
\end{verbatim}
\subsubsection{Description}
Retrieve the graphic limits associated with a number of PVs.
\subsubsection{Parameters}
\begin{description}
\PVITEM
%%
\item[lowLimit] \mxl{} column vector of the PV's low graphic limit.
\item[hiLimit]  \mxl{} column vector of the PV's high graphic limit.
\end{description}

\vspace*{\fill}
\pbrk
\subsection{lcaGetStatus}
\subsubsection{Calling Syntax}
\begin{verbatim}
[severity, status, timestamp] = lcaGetStatus(pvs)
\end{verbatim}
\subsubsection{Description}
Retrieve the alarm severity and status of a number of PVs along
with their timestamp.
\subsubsection{Parameters}
\begin{description}
\PVITEM
%%
\item[severity] \mxl{} column vector of the alarm severities.
\item[status] \mxl{} column vector of the alarm status.
\item[timestamp] \mxl{} \ita{complex} column vector holding the
PV \hyperref[ref]{timestamps}{timestamps (see }{ about the timestamp format)}{tsformat}.
\end{description}


\vspace*{\fill}
\pbrk
\subsection{lcaGetPrecision}
\subsubsection{Calling Syntax}
\begin{verbatim}
precision = lcaGetPrecision(pvs)
\end{verbatim}
\subsubsection{Description}
Retrieve the precision of a number of PVs.
\subsubsection{Parameters}
\begin{description}
\PVITEM
%%
\item[precisiom] \mxl{} column vector of the PV's precision.
\end{description}

\vspace*{\fill}
\pbrk
\subsection{lcaGetUnits}
\subsubsection{Calling Syntax}
\begin{verbatim}
units = lcaGetUnits(pvs)
\end{verbatim}
\subsubsection{Description}
Retrieve the engineering units of a number of PVs.
\subsubsection{Parameters}
\begin{description}
\PVITEM
%%
\item[units] \mxl{} column vector (on \matlab: \ita{cell-} matrix)
of strings holding the PV EGUs.
\end{description}

\vspace*{\fill}
\pbrk
\subsection{lcaGetRetryCount, lcaSetRetryCount}
\label{retrycnt}
\subsubsection{Calling Syntax}
\begin{verbatim}
currentRetryCount = lcaGetRetryCount()
lcaSetRetryCount(newRetryCount)
\end{verbatim}
\subsubsection{Description}
Retrieve / set the \ezca{} library retryCount parameter (consult the \ezca{}
documentation for more information). The retry count multiplied by the
\hyperref[ref]{timeout parameter}{timeout parameter (see }{ )}{timeout}
determines the maximum time the interface waits for connections and data
transfers, respectively.

\vspace*{\fill}
\pbrk
\subsection{lcaGetTimeout, lcaSetTimeout}
\label{timeout}
\subsubsection{Calling Syntax}
\begin{verbatim}
currentTimeout = lcaGetTimeout()
lcaSetTimeout(newTimeout)
\end{verbatim}
\subsubsection{Description}
Retrieve / set the \ezca{} library timeout parameter (consult the \ezca{}
documentation for more information). Note that the semantics of the
timeout parameter has changed with \sca{} version 3. The library no longer
pends for CA activity in multiples of this timeout value but returns
control to \scilab{} as soon as the underlying CA request completes.

However, \sca{} checks for ``Ctrl-C'' key events every time
(and only when) the timeout expires. Hence, it is convenient to
choose a value $<1$s.

The {\em maximal} time spent waiting for connections and/or data equals the
timeout multiplied by the
\hyperref[ref]{retry count}{retry count (see }{ )}{retrycnt}.

\vspace*{\fill}
\pbrk
\subsection{lcaDebugOn, lcaDebugOff}
\subsubsection{Calling Syntax}
\begin{verbatim}
lcaDebugOn()
lcaDebugOff()
\end{verbatim}
\subsubsection{Description}
Switch the \ezca{} library's debugging facility on and off, respectively.

\vspace*{\fill}
\pbrk
\subsection{lcaSetSeverityWarnLevel}
\label{swarnlevel}
\subsubsection{Calling Syntax}
\begin{verbatim}
lcaSetSeverityWarnLevel(newlevel)
\end{verbatim}
\subsubsection{Description}
Set the warning threshold for \com{lcaGet()} operations.
A warning message is printed when retrieving a PV with a severity bigger
or equal to the warning level. Supported values are \sevrange{} (No alarm,
minor alarm, major alarm, invalid alarm). The initial/default value is 3.

If a value $>=10$ is passed, the threshold for refusing to read the \com{.VAL}
field of PVs with an \ita{INVALID} severity can be changed. The rejection
can be switched off completely by passing 14 ($= 10 + INVALID\_ALARM + 1$)
or made more sensitive by passing a value of less than 13 ($=10 + INVALID\_ALARM$),
the default.

\vspace*{\fill}
\pbrk
\subsection{lcaDelay}
\label{lcadelay}
\subsubsection{Calling Syntax}
\begin{verbatim}
lcaDelay(timeout)
\end{verbatim}
\subsubsection{Description}
Delay execution of \scilab{} or \matlab{} for the specified time
to handle channel access activity (monitors). 
{\em Using this call is not needed under EPICS-3.14} since monitors
are transparently handled by separate threads. These ``worker threads''
receive data from CA on monitored channels ``in the background'' while
\scilab{}/\matlab{} are processing arbitrary calculations.
You only need to either poll the library for the data being ready using the
\comref{lcaNewMonitorValue()}{lcanewmonitorvalue}) routine
or block for data becoming available using
\comref{lcaNewMonitorWait}{lcanewmonitorwait}.

\subsubsection{Parameters}
\begin{description}
\item[timeout]
A timeout value in seconds.
\end{description}
%\subsubsection{Examples}
%\begin{verbatim}
%lcaDelay(2.0)
%\end{verbatim}

\pbrkf
\section{Building and Using \sca{}}
\subsection{Build}
{\em NOTE: If the binaries distributed with \sca{} work for you then there is
no need to build anything. If you want/need to build your own version then read on
otherwise proceed to \hyperref{Using \sca}{Subsection~}{}{usingsca}.}

\sca{} comes with a `\com{configure}' subdirectory and Makefiles conforming to the
EPICS build system. Following a configuration step which involves editing
two small files, '\com{make}' is executed to install the generated libraries
and scripts.

Prior to invoking the \scilab{} or \matlab{} application, the system
must be properly set up in order for the applications to locate the
\sca{} and channel access libraries.

\subsubsection{Prerequisites}
\sca{} needs an EPICS BASE installation that was built \ita{with shared
libraries enabled}%
\footnote{
To be precise: only \sca{} needs to be a shared library (but must then have
EPICS BASE linked in); we recommend an EPICS BASE installation with
shared libraries enabled because modifications to the EPICS Makefiles
are required to build a shared \sca{} library that is linked against
static versions of EPICS BASE.}%
.
The main reason being that \matlab's \ita{mex} files cannot
have multiple entry points. Hence, when statically linking multiple \ita{mex}
files against \ita{ezca, ca, Com} etc. multiple copies of those libraries
would end up in the running \matlab{} application with possible adverse
effects. It should be possible to build and use the \scilab{} interface
with static libraries --- minor tweaks to the Makefiles might be necessary.

\sca{} has been tested with \matlab-6.5, \matlab-7.0 and \scilab-2.7 .. \scilab-4.1
under a variety of EPICS releases up to 3.14.9 on \ita{linux-x86},
\ita{linux-ppc}, \ita{solaris-sparc-gnu},
\ita{linux-x86\_64}, \ita{solaris-sparc}, \ita{solaris-sparc64}
and \windoze%
\footnote{
Note that not all possible combinations have been tested with the latest \sca{} release
but rather the latest versions of the respective components on the platforms
that are in the distribution.}%
.

Note that the binary distribution of \sca{} usually ships with the
necessary EPICS base libraries so there is no need to download anything
besides \sca.
%The \ita{linux}
%and \ita{solaris} versions have these libraries built-in, the \windoze
%version comes with the appropriate DLLs.

\pbrkf
\subsubsection{Configuration}
Two files, `\com{configure/CONFIG}' and `\com{configure/RELEASE}' need to
be adapted to the user's installation:
\begin{description}
\item[\tt CONFIG:] A handful of configuration parameters must be defined
in this file.
\begin{description}
\item[\tt MAKEFOR:] Setting the \com{MAKEFOR} variable
determines the target application program the interface library is
built for. Valid settings are \com{MAKEFOR=SCILAB} or \com{MAKEFOR=MATLAB}.
Any setting other than \com{MATLAB} is treated like \com{SCILAB}.

\item[\tt CONFIG\_USE\_CTRLC:] Set this to {\tt YES} or {\tt NO} to enable
or disable, respectively, code for handling ``Ctrl-C'' keystroke sequences.
When enabled, \sca{} operations ({\em except for} \com{lcaDelay}) may be aborted
by hitting ``Ctrl-C''.
Note that \sca{} polls for an ``abort condition'' with a granularity of the
\hyperref[ref]{\ezca{} timeout parameter}{timeout parameter (see }{ )}{timeout}.
Unfortunately, neither \matlab{} nor \scilab{} feature a documented API for handling
Ctrl-C events and therefore Ctrl-C support --- the implementation using undocumented
features of \scilab{} and \matlab{} --- must be considered ``experimental''
i.e., it might cause problems on certain operating system and/or \scilab/\matlab{}
versions.

\item[\tt INSTALL\_LOCATION:] Set this variable to install in a location
different from the \sca{} top directory.

{\em NOTE: This method has been
deprecated. Use \com{INSTALL\_LOCATION\_APP} in the \com{RELEASE} file
instead.}
\end{description}
%
%
\item[\tt RELEASE:] In this file, paths to the EPICS base (`\com{EPICS\_BASE}'
variable) and \scilab{} (`\com{SCILABDIR}' variable) or \matlab{} (`\com{MATLABDIR}'
variable) installations must be specified.

Under \windoze{}, an additional variable
`\com{MATLIB\_SUBDIR}' must be set directing the build process to select the correct
\com{libmx.lib} and \com{libmex.lib} library variants. The setting of this variable
is compiler dependent.
\begin{description}
\item[\tt INSTALL\_LOCATION\_APP=<path>] \com{<path>} defining the
install location of the \sca{} package. If unset, the \sca{} \com{TOP}
directory will be used.
\item[\tt MATLABDIR=<path>] \com{<path>} defining the \matlab{} installation
directory where the `\com{extern}' subdirectory can be found (e.g.
\com{/opt/matlabR14beta2}).
\item[\tt SCILABDIR=<path>] \com{<path>} defining the \scilab{} installation
directory where the `\com{routines}' subdirectory can be found (e.g.
\com{/usr/lib/scilab-2.7}).
\item[\tt MATLIB\_SUBDIR=<pathelem>] \com{<pathelem>} chosing the subdirectory
corresponding to the C-compiler that is used for the build. (e.g. 
\com{win32/microsoft/msvc60} for the microsoft visual c++ 6.0 compiler).
The \com{libmex.lib} and \com{libmx.lib} files for the applicable compiler
are found there.
\end{description}
\end{description}
Any irrelevant variables (such as \com{MATLABDIR} if \com{MAKEFOR=SCILAB})
are ignored.

Note that the EPICS build system has problems with path names containing
white space as they are commonly used on \windoze. Although I have
tried to work around this, you still might encounter problems. I found
that setting the environment variable \com{MATLAB} to point to the \matlab{}
directory helped (\ita{cygwin}). It is best to avoid white space in path
names, however.

\subsubsection{Building \sca}
After setting the
`\com{EPICS\_HOST\_ARCH}' environment variable, \ita{GNU} make is invoked from the \sca{} top 
directory. Note that the compiler toolchain must be found in the system
\com{PATH}\footnote{
Under \windoze, the \ita{msvc} compiler features a \com{.BAT} file for
setting up the necessary environment.
}.

\pbrk
\subsection{Using \sca{}}
\label{usingsca}
\sca{} consists of a set of shared libraries which in turn reference
other shared libraries (most notably the channel access client libraries
from EPICS BASE). It is of \ita{crucial importance} that the operating
system locates the correct versions at \ita{run-time} (i.e. the same versions
\sca{} was \ita{linked} against). Otherwise, the run-time linker/loader
could fail to load the required objects --- leaving the user (expecially
in \matlab) at the prompt with obscure error messages.

Under \ita{linux} or \ita{solaris}, the \com{LD\_LIBRARY\_PATH} environment
variable or the \com{ld.so.conf / ldconfig} facility are used to point
to the executable shared libraries (located in \com{lib/<arch>}).
If using a binary distribution, the \com{PATH} variable also should point
to \com{bin/<arch>} so that the \ita{CA repeater} executable is found.
If you build from source using your own EPICS base installation then
we assume that locating the \ita{CA repeater} has already been taken
care of.

Under \windoze{}, the \com{PATH} environment variable must point to the
correct EPICS BASE and \sca{} DLLs (located in \com{bin/<arch>}).

Note that the paths to the correct EPICS BASE and \sca{} shared libraries
must be set up \ita{prior} to starting the \scilab{} or \matlab{} application.
It is usually not possible to change the system search
path from within an application.

Possible problems could occur because
\begin{itemize}
\item third party setup scripts modify \com{LD\_LIBRARY\_PATH / PATH}.
\item your EPICS BASE was not built with shared libraries but shared libraries
of a different release are found somewhere.
\item a ``innocent-looking'' directory present in \com{LD\_LIBRARY\_PATH}
before EPICS BASE actually contains shared libraries of an older release.
\item Note that the system search path is \ita{not} identical with the
\matlab{} path --- changing the \matlab{} path (from withing \matlab)
has no influence on the system search path.
\end{itemize}

\subsubsection{Using \sca{} with \scilab}
Set up the shared library search path (as described above) and start
\scilab. Run the \com{\sca.sce} script which was generated by the build process
(\com{<\sca-top>/bin/<arch>/\sca.sce}) to load the \sca{} interface.
The script also adds to the \com{\%helps} variable making on-line
help available.

The script can be installed at any convenient location --- the lines
setting up the \com{\%helps} path need to be adapted in this case.

It is also possible to permanently link \sca{} to \scilab{}. Consult
the \scilab{} documentation for more information.

\subsubsection{Using \sca{} with \matlab}
Every entry point / \com{lcaXXX} routine 
is contained in a separate
shared object (AKA ``\com{mex}-'') file in the \com{<\sca-top>/bin/<arch>/labca}%
\footnote{
the extra subdirectory was added in order for the \matlab{} \com{help} command
to easily locate the \com{Contents.m} file when the user types \com{help labca}.
} directory
which must be added to the \matlab{} search path.
Note that this is in \ita{addition}
to setting the system library search path which must be performed prior
to starting \matlab{} (see previous section).
All necessary objects and libraries are transparently loaded simply
by invoking any of the entry point routines. Note that on-line help
files are also installed to be located automatically.
\end{document}
